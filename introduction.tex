%!TEX root = thesis.tex
\chapter{Introduction}

\subsection{Topic covered by the project}
Everything actually started with a simple mathematical theory by John Von Neumann \cite{von1966theory} and the first example of practical malware is written by Bob Thomas at BBN, and it was called Creeper 

The malware is abbreviation of malicious software. It could be any piece of code which is defined malicious. There is no formal definition of malicious, it could be some software advertise without any contest or it could be self-producing code piece which aim to distribute itself and steal your private information, and it turned an arm race between power holders today.

With development of the first malware, their counter software are created and anti malware software have evolved with them so far. In this race, malware authors are always one step further, because of security's nature. This race between black and white side raised the bar above. The motivation of the information amount and severity influence both today, and that information can be sometimes vital. 

\subsection{Keywords}
Security, Malware Design, Cache Oriented Polymorphism, Cache Coherency, Malware Evasion, Code Obfuscation

\subsection{Problem description}


\subsection{Justification, motivation and benefits}
If malware designing is superficially considered, you could fall in the usual fallacy that it is not beneficial, and maybe it is malicious. However, if we can design it, there is always a more skillful malicious author who might already abuse this vulnerability on the black side of the moon. The duty we are actually obligated to discover these vulnerabilities and design countermeasure against them. In this way, our blessed motivation is finding possible vulnerabilities, and mitigate or eliminate their risk. Otherwise; if we are lucky, we might detect these zero time vulnerability attacks, yet it could be too late to fix and analyze them. Besides, for example, some of the most sensational and beneficial papers(\cite{moser2007limits},\cite{cavallaro2008limits},\cite{egele2012survey}) are criticizing malware with designing them as like as we do, and their values over computer security are undoubted today. 

In short, we are building brakes. Sometimes they are that thing slow us down, or sometimes they even stop us in our tracks. However, they are actually there to enable us to go faster and secure.

\subsection{Research questions}\label{research:questions}

\begin{enumerate}
	\item How can an an obfuscation method, which exploits caches to conceal itself from memory scanning systems, be designed for tightly coupled, multiprocessor systems?
	\item How can we design an attack to bypass snoop cache coherency?
	\item How can we execute deobfuscated code on the Harvard Architecture without leakage to upper memories?
\end{enumerate}


\subsection{Planned contributions}
This Master thesis is looking for better understanding on cache usage with obfuscation techniques. Especially, It will try to show how possible to abuse tightly coupled systems accessing and how durable recent detection kits. It is quite unique work which we have to consider on the future. Ultimate goal is to eliminate any uncertainties which detection methods encounter with concurrent memory accessing.

\section{Choice of methods}
	This thesis will use a technical approach to the problem. It will use quantitative and model building approach. The methodology consist 8 circular step which are, asking question, building new hypothesis, planning methods, developing software, preparing generic testbed, testing, analyzing, reporting results. In addition to this, the large portion of the time will be given researching related topics, based architecture, and learning tools and technologies. Consequently, the accuracy of the thesis is lies on the proper scientific methodology.

	\todo[inline]{ circular diagram will be Drawn here}


	To address the first question, designing proper malware evasion technique with concurrent and parallel architecture have not been researched well so for, therefore; it will lies on so much experiment and we might have to reflection of other evasion and obfuscation attacks analogy.  We will chose several known malware, which is on air, to  evade them during testing part. We will strong probably use kernel modules and android operating systems for testing bed, however we could linux OS without android layer to simplify and closure test period. In testing step, we will use several anti-virus system such as Avast, Comodo, Norton and compare the result before evasion and after evasion. Testing period might me include mathematical proof depending on evasion or obfuscation method.\todo{Consider involvement of this line during project.} At the end of the each hypothesis` result, which is mean the method for concealing malware, will be reported properly. Each method will be another hypothesis, so we could be proof whether multiple or none successful hypotheses, yet the failed hypothesis could be crucial. There could be also many result which are too baroque to proof them or explain their result, these cases could be observed on further work. For semantic knowledge, we could try to show relationship between evasion methods and hybrid approaches.

	If we can find successful hypothesis for first question, we observe them in second question. Second question is depend on the first question's answer. Second questions methodology is actually exactly same circular. It start with defense hypothesis against evasion method. Dynamic and static detection methods must be both considered. The development of the counter algorithms could be proved mathematically, but it can be quite baroque to formulate it. In order to prove it, we could design evasion and detection methods` Turing model, however; the main approach of our testing is involved with experimental solution. we will develop planned algorithms prototype. Proper test bed could be provided with lots of malware species.. We will test our algorithms` prototype with concealed or obfuscated viruses. If it is really necessary, we could prepare also control groups to prove trustworthiness of method, then we have to record result without any intervention. For semantic knowledge, we could try to show relationship between detection methods and hybrid approaches.
	\todo{depending on the found evasion technique, methodology could be shaped again}
	\newpage

	The last question is a matter of measuring and analyzing algorithms complexity. It is totally mathematical scientific methodology. We have to analyze worst, best and average complexity rate. There are also several Quantitative approaches like measuring resource usage. It could be efficient some system like network which there are lots of uncertainties in.


\section{Thesis Outline}
	This section provide a brief summary listing of the content presented in this thesis. The listing is based on chapters, where each chapters and its content is described. First the related works and background studies are presented. Then, our designs is presented as well as further work and conclusion in a sequence. At the end of the thesis, we attached the simulation code which we wrote and the experiments results by Booksim v2.0 to appendix.
\begin{itemize}
\item Chapter 2
\item Chapter 3
\item Chapter 4
\item Chapter 5
\item Chapter 6
\item Chapter 7
\item Appendix A
\item Appendix B
\end{itemize}
