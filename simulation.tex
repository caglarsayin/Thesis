\chapter{Cache Memory Simulation}
\begin{verbatim}
__author__ = 'caglar'
import random


class Memory(object):
    def __init__(self):
        self.memory = {}

    def __getitem__(self, item):
        try:
            return self.memory[item]
        except KeyError:
            return 0

    def __setitem__(self, key, value):
        self.memory[key] = value


class Cache(object):
    def __init__(self, size=32768, block_size=64, sets=2):
        self.size = size
        self.block_size = block_size
        self.sets = sets
        self.addr_size = 32
        self.cache = list()
        self.line = size / (sets * block_size)
        self.block_addr_len = self.__calculate_addr_len(self.block_size)
        self.line_addr_len = self.__calculate_addr_len(self.line)
        self.tag_addr_len = self.addr_size - self.block_addr_len - self.line_addr_len
        self.super = None
        self.__build()

    def __repr__(self):
        return str(self.cache.__len__())

    def __getitem__(self, address):
        query = self.toquery(address)
        return self.read_request(query)

    def __setitem__(self, address, value):
        self.write_request(address, value)

    def setsuper(self,super):
        self.super = super

    def __build(self):
        for i in range(self.line):
            self.cache.append(CacheLine(i, self.block_size, self.sets))

    def __calculate_addr_len(self, size):
        i = 0
        result = 1
        while size > result:
            result <<= 1
            i += 1
        return i

    def toquery(self, address):
        mask = (1 << 32) - 1
        query = {"address": address,
                 "tag": address >> (self.addr_size - self.tag_addr_len),
                 "index": (address >> (self.block_addr_len)) << ((self.addr_size - self.line_addr_len) & mask) >> (
                     self.addr_size - self.line_addr_len),
                 "block": ((address << (self.addr_size - self.block_addr_len)) & mask) >> (
                     self.addr_size - self.block_addr_len)}
        return query

    def isvalid(self, query):
        for i in range(self.cache[query["index"]].sets):
            if (self.cache[query["index"]].lines[i]["tag"] is query["tag"]) and (self.cache[query["index"]].lines[i]['valid'] is True):
                self.cache[query["index"]].used = i
                return self.cache[query["index"]].lines[i]
        return False

    def read_request(self, query):
        line = self.isvalid(query)
        if line:
            return self.__read_hit(line, query)
        else:
            return self.__read_miss(query)

    def __read_hit(self, line, query):
        return line["data"][query["block"]]

    def __read_miss(self, query):
        self.request_line(query) #proof it later.
        return self.read_request(query)


    def write_request(self, address, value):
        query = self.toquery(address)
        line = self.isvalid(query)
        if line is not -1:
            return self.__write_hit(line, query, value)
        else:
            return self.__write_miss(query)

    def __write_hit(self, line, query, value):
        line["data"][query["block"]] = value
        line["dirty"] = True

    def __write_miss(self, query):
        pass

    def request_line(self, query):
        line = query["address"] - query["block"]
        rline = {"tag": query["tag"], "data": bytearray(self.block_size)}
        for i in range(self.block_size):
            rline["data"][i] = self.super[line+i]
        self.cache[query["index"]].fill_line(rline)


class CacheLine(object):
    #CacheLine is a Class for defining each cache blocks in cache hierarchy
    #During initialization it build a static block of given sets sized array with given block size as byte
    #It has only one method to fill remote block in.
    def __init__(self, index, size, sets):
        self.size = size
        self.sets = sets
        self.index = index
        self.lines = list()
        self.__build()
        self.used = 0

    def __repr__(self):
        return repr(self.lines)

    def __build(self):
        #Initialize the first view of the cache.
        #It will invoke build_line() for each line of the cache.
        for i in range(self.sets):
            self.lines.append(self.__build_line())

    def __build_line(self):
        #Build a line with default variables.
        #data is a byte array which is given length with size
        data = bytearray(self.size)
        return {"tag": 0, "valid": False, "dirty": False, "used": False, "data": data}

    def mark_used(self, line):
        self.used = line

    def give_replaceable(self):
        if self.sets == 1:
            return 0
        else:
            line = random.randint(0, self.sets - 1)
            if self.used != line:
                return line
            else:
                return (line + 1) % self.sets

    def fill_line(self, rline):
        # It fills the corresponding block from the upper layer into the line .
        # Input "rline" is abbreviation of remote line formed as {tag, data}
        # As default, it uses Not recently used, random replacement Policy
        i = self.give_replaceable()
        self.lines[i]["tag"] = rline["tag"]
        self.lines[i]["valid"] = True
        self.lines[i]["dirty"] = False
        self.lines[i]["data"] = rline["data"]
\end{verbatim}